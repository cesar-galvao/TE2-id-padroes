% Options for packages loaded elsewhere
\PassOptionsToPackage{unicode}{hyperref}
\PassOptionsToPackage{hyphens}{url}
\PassOptionsToPackage{dvipsnames,svgnames,x11names}{xcolor}
%
\documentclass[
  a4paperpaper,
]{article}

\usepackage{amsmath,amssymb}
\usepackage{iftex}
\ifPDFTeX
  \usepackage[T1]{fontenc}
  \usepackage[utf8]{inputenc}
  \usepackage{textcomp} % provide euro and other symbols
\else % if luatex or xetex
  \ifXeTeX
    \usepackage{mathspec} % this also loads fontspec
  \else
    \usepackage{unicode-math} % this also loads fontspec
  \fi
  \defaultfontfeatures{Scale=MatchLowercase}
  \defaultfontfeatures[\rmfamily]{Ligatures=TeX,Scale=1}
\fi
\usepackage{lmodern}
\ifPDFTeX\else  
    % xetex/luatex font selection
\fi
% Use upquote if available, for straight quotes in verbatim environments
\IfFileExists{upquote.sty}{\usepackage{upquote}}{}
\IfFileExists{microtype.sty}{% use microtype if available
  \usepackage[]{microtype}
  \UseMicrotypeSet[protrusion]{basicmath} % disable protrusion for tt fonts
}{}
\makeatletter
\@ifundefined{KOMAClassName}{% if non-KOMA class
  \IfFileExists{parskip.sty}{%
    \usepackage{parskip}
  }{% else
    \setlength{\parindent}{0pt}
    \setlength{\parskip}{6pt plus 2pt minus 1pt}}
}{% if KOMA class
  \KOMAoptions{parskip=half}}
\makeatother
\usepackage{xcolor}
\usepackage[top=30mm,left=30mm,right=30mm,heightrounded]{geometry}
\setlength{\emergencystretch}{3em} % prevent overfull lines
\setcounter{secnumdepth}{-\maxdimen} % remove section numbering
% Make \paragraph and \subparagraph free-standing
\ifx\paragraph\undefined\else
  \let\oldparagraph\paragraph
  \renewcommand{\paragraph}[1]{\oldparagraph{#1}\mbox{}}
\fi
\ifx\subparagraph\undefined\else
  \let\oldsubparagraph\subparagraph
  \renewcommand{\subparagraph}[1]{\oldsubparagraph{#1}\mbox{}}
\fi

\usepackage{color}
\usepackage{fancyvrb}
\newcommand{\VerbBar}{|}
\newcommand{\VERB}{\Verb[commandchars=\\\{\}]}
\DefineVerbatimEnvironment{Highlighting}{Verbatim}{commandchars=\\\{\}}
% Add ',fontsize=\small' for more characters per line
\usepackage{framed}
\definecolor{shadecolor}{RGB}{241,243,245}
\newenvironment{Shaded}{\begin{snugshade}}{\end{snugshade}}
\newcommand{\AlertTok}[1]{\textcolor[rgb]{0.68,0.00,0.00}{#1}}
\newcommand{\AnnotationTok}[1]{\textcolor[rgb]{0.37,0.37,0.37}{#1}}
\newcommand{\AttributeTok}[1]{\textcolor[rgb]{0.40,0.45,0.13}{#1}}
\newcommand{\BaseNTok}[1]{\textcolor[rgb]{0.68,0.00,0.00}{#1}}
\newcommand{\BuiltInTok}[1]{\textcolor[rgb]{0.00,0.23,0.31}{#1}}
\newcommand{\CharTok}[1]{\textcolor[rgb]{0.13,0.47,0.30}{#1}}
\newcommand{\CommentTok}[1]{\textcolor[rgb]{0.37,0.37,0.37}{#1}}
\newcommand{\CommentVarTok}[1]{\textcolor[rgb]{0.37,0.37,0.37}{\textit{#1}}}
\newcommand{\ConstantTok}[1]{\textcolor[rgb]{0.56,0.35,0.01}{#1}}
\newcommand{\ControlFlowTok}[1]{\textcolor[rgb]{0.00,0.23,0.31}{#1}}
\newcommand{\DataTypeTok}[1]{\textcolor[rgb]{0.68,0.00,0.00}{#1}}
\newcommand{\DecValTok}[1]{\textcolor[rgb]{0.68,0.00,0.00}{#1}}
\newcommand{\DocumentationTok}[1]{\textcolor[rgb]{0.37,0.37,0.37}{\textit{#1}}}
\newcommand{\ErrorTok}[1]{\textcolor[rgb]{0.68,0.00,0.00}{#1}}
\newcommand{\ExtensionTok}[1]{\textcolor[rgb]{0.00,0.23,0.31}{#1}}
\newcommand{\FloatTok}[1]{\textcolor[rgb]{0.68,0.00,0.00}{#1}}
\newcommand{\FunctionTok}[1]{\textcolor[rgb]{0.28,0.35,0.67}{#1}}
\newcommand{\ImportTok}[1]{\textcolor[rgb]{0.00,0.46,0.62}{#1}}
\newcommand{\InformationTok}[1]{\textcolor[rgb]{0.37,0.37,0.37}{#1}}
\newcommand{\KeywordTok}[1]{\textcolor[rgb]{0.00,0.23,0.31}{#1}}
\newcommand{\NormalTok}[1]{\textcolor[rgb]{0.00,0.23,0.31}{#1}}
\newcommand{\OperatorTok}[1]{\textcolor[rgb]{0.37,0.37,0.37}{#1}}
\newcommand{\OtherTok}[1]{\textcolor[rgb]{0.00,0.23,0.31}{#1}}
\newcommand{\PreprocessorTok}[1]{\textcolor[rgb]{0.68,0.00,0.00}{#1}}
\newcommand{\RegionMarkerTok}[1]{\textcolor[rgb]{0.00,0.23,0.31}{#1}}
\newcommand{\SpecialCharTok}[1]{\textcolor[rgb]{0.37,0.37,0.37}{#1}}
\newcommand{\SpecialStringTok}[1]{\textcolor[rgb]{0.13,0.47,0.30}{#1}}
\newcommand{\StringTok}[1]{\textcolor[rgb]{0.13,0.47,0.30}{#1}}
\newcommand{\VariableTok}[1]{\textcolor[rgb]{0.07,0.07,0.07}{#1}}
\newcommand{\VerbatimStringTok}[1]{\textcolor[rgb]{0.13,0.47,0.30}{#1}}
\newcommand{\WarningTok}[1]{\textcolor[rgb]{0.37,0.37,0.37}{\textit{#1}}}

\providecommand{\tightlist}{%
  \setlength{\itemsep}{0pt}\setlength{\parskip}{0pt}}\usepackage{longtable,booktabs,array}
\usepackage{calc} % for calculating minipage widths
% Correct order of tables after \paragraph or \subparagraph
\usepackage{etoolbox}
\makeatletter
\patchcmd\longtable{\par}{\if@noskipsec\mbox{}\fi\par}{}{}
\makeatother
% Allow footnotes in longtable head/foot
\IfFileExists{footnotehyper.sty}{\usepackage{footnotehyper}}{\usepackage{footnote}}
\makesavenoteenv{longtable}
\usepackage{graphicx}
\makeatletter
\def\maxwidth{\ifdim\Gin@nat@width>\linewidth\linewidth\else\Gin@nat@width\fi}
\def\maxheight{\ifdim\Gin@nat@height>\textheight\textheight\else\Gin@nat@height\fi}
\makeatother
% Scale images if necessary, so that they will not overflow the page
% margins by default, and it is still possible to overwrite the defaults
% using explicit options in \includegraphics[width, height, ...]{}
\setkeys{Gin}{width=\maxwidth,height=\maxheight,keepaspectratio}
% Set default figure placement to htbp
\makeatletter
\def\fps@figure{htbp}
\makeatother

\usepackage{fvextra}
\usepackage{bbm}
\usepackage[auth-lg]{authblk}
\DefineVerbatimEnvironment{Highlighting}{Verbatim}{breaklines,commandchars=\\\{\}}
\DefineVerbatimEnvironment{OutputCode}{Verbatim}{breaklines,commandchars=\\\{\}}
\makeatletter
\@ifpackageloaded{caption}{}{\usepackage{caption}}
\AtBeginDocument{%
\ifdefined\contentsname
  \renewcommand*\contentsname{Índice}
\else
  \newcommand\contentsname{Índice}
\fi
\ifdefined\listfigurename
  \renewcommand*\listfigurename{Lista de Figuras}
\else
  \newcommand\listfigurename{Lista de Figuras}
\fi
\ifdefined\listtablename
  \renewcommand*\listtablename{Lista de Tabelas}
\else
  \newcommand\listtablename{Lista de Tabelas}
\fi
\ifdefined\figurename
  \renewcommand*\figurename{Figura}
\else
  \newcommand\figurename{Figura}
\fi
\ifdefined\tablename
  \renewcommand*\tablename{Tabela}
\else
  \newcommand\tablename{Tabela}
\fi
}
\@ifpackageloaded{float}{}{\usepackage{float}}
\floatstyle{ruled}
\@ifundefined{c@chapter}{\newfloat{codelisting}{h}{lop}}{\newfloat{codelisting}{h}{lop}[chapter]}
\floatname{codelisting}{Listagem}
\newcommand*\listoflistings{\listof{codelisting}{Lista de Listagens}}
\makeatother
\makeatletter
\makeatother
\makeatletter
\@ifpackageloaded{caption}{}{\usepackage{caption}}
\@ifpackageloaded{subcaption}{}{\usepackage{subcaption}}
\makeatother
\ifLuaTeX
\usepackage[bidi=basic]{babel}
\else
\usepackage[bidi=default]{babel}
\fi
\babelprovide[main,import]{portuguese}
% get rid of language-specific shorthands (see #6817):
\let\LanguageShortHands\languageshorthands
\def\languageshorthands#1{}
\ifLuaTeX
  \usepackage{selnolig}  % disable illegal ligatures
\fi
\usepackage{bookmark}

\IfFileExists{xurl.sty}{\usepackage{xurl}}{} % add URL line breaks if available
\urlstyle{same} % disable monospaced font for URLs
\hypersetup{
  pdftitle={Lista 4},
  pdfauthor={César A. Galvão - 190011572},
  pdflang={pt},
  colorlinks=true,
  linkcolor={blue},
  filecolor={Maroon},
  citecolor={Blue},
  urlcolor={Blue},
  pdfcreator={LaTeX via pandoc}}

\title{Lista 4}
\author{César A. Galvão - 190011572}
\date{}

\begin{document}
\maketitle

\renewcommand*\contentsname{Índice}
{
\hypersetup{linkcolor=}
\setcounter{tocdepth}{2}
\tableofcontents
}
\newpage{}

\section{Questão 11}\label{questuxe3o-11}

Pesquisar funções disponíveis em pacotes R para classificação utilizando
a função logística. Apresentar um pequeno exemplo do uso das funções.
Destacar vantagens e desvantagens em relação aos pacotes de Modelos
Lineares Generalizados apresentados em aula.

Exemplos de pacotes para classificação no R: \texttt{caret},
\texttt{class}, \texttt{mlpack.}

\begin{center}\rule{0.5\linewidth}{0.5pt}\end{center}

~

A seguir são apresentadas as técnicas utilizadas em sala de aula e as
funções dos pacotes \texttt{caret}. Não foi possível instalar o pacote
\texttt{mlpack} e o pacote \texttt{class} não compreende funções para
regressão logística.

A regressão logística é um Modelo Linear Generalizado, que pode ser
descrito como

\[
g\left[ E(Y_i) \right] = \beta_0 + \beta_1 X_{1i} + \ldots + \beta_p X_{pi}, \quad Y_i \overset{i.i.d.}{\sim} FE\left( g\left[ E(Y_i) \right], \sigma^2 \right),
\] ~

\noindent em que \(g\) é a função de ligação sobre o preditor linear
\(g\left[ E(Y_i) \right] = \mathbf{X}\boldsymbol{\beta}\). A função de
ligação mais comum para o modelo logístico é a função logit, dada por
\(g(\pi) = \log\left( \frac{\pi}{1-\pi} \right)\),
\(\pi_i = P(Y_i = 1 | \mathbf{X} = \mathbf{x}_i)\), e considera-se
\(Y_i \sim \text{Bernoulli}\left( p(\mathbf{x}|\omega_1) \right)\).

No contexto de classificação binária, temos que

\[
\frac{p(\mathbf{x}|\omega_1)}{1-p(\mathbf{x}|\omega_1)} = \frac{p(\mathbf{x}|\omega_1)}{p(\mathbf{x}|\omega_2)} = \exp\left( \mathbf{X}\boldsymbol{\beta} \right),
\] ~

\noindent considerando
\(\mathbf{X} = (\mathbbm{1}^\top, \mathbf{X}^*)\), \(\mathbf{X}^*\) a
matriz de covariáveis.

A decisão de alocação de \(\mathbf{x}_i\) a \(\omega_1\) ocorre se
\(p(\mathbf{x}|\omega_1)> k\), constante que comumente é \(0,5\).

~

Os modelos apresentados em aula compreendem regressão logística,
múltipla, politômica e politômica ordenada. Para isso, diversos pacotes
são utilizados como \texttt{stats}, \texttt{mlpack} e \texttt{VGAM}.

Para os modelos dicotômicos, são apresentados resultados de seleção de
variáveis e medidas diagnósticas como medidas de influências, qualidade
de ajuste com \(G^2\), razão de verossimilhança e teste de Hosmer e
Lemeshow.

Enquanto a implementação via ferramentas do pacote \texttt{stats} seja
factível, o pacote \texttt{caret} apresenta um \emph{framework}
consistente para o fluxo de modelagem.

Por exemplo, o bloco a seguir apresenta a partição de uma base de dados
em treino e teste com 80\% dos dados destinados ao treino. O ajuste do
modelo e a matriz de confusão são realizados funções do próprio pacote,
mas é utilizado o \texttt{predict()} genérico do pacote \texttt{stats}:

~

\begin{Shaded}
\begin{Highlighting}[]
\FunctionTok{data}\NormalTok{(iris)}

\NormalTok{iris}\SpecialCharTok{$}\NormalTok{Class }\OtherTok{\textless{}{-}} \FunctionTok{ifelse}\NormalTok{(iris}\SpecialCharTok{$}\NormalTok{Species }\SpecialCharTok{==} \StringTok{"versicolor"}\NormalTok{, }\DecValTok{1}\NormalTok{, }\DecValTok{0}\NormalTok{)}

\FunctionTok{set.seed}\NormalTok{(}\DecValTok{123}\NormalTok{)}
\NormalTok{trainIndex }\OtherTok{\textless{}{-}} \FunctionTok{createDataPartition}\NormalTok{(iris}\SpecialCharTok{$}\NormalTok{Class, }\AttributeTok{p =}\NormalTok{ .}\DecValTok{8}\NormalTok{, }\AttributeTok{list =} \ConstantTok{FALSE}\NormalTok{, }\AttributeTok{times =} \DecValTok{1}\NormalTok{)}
\NormalTok{trainData }\OtherTok{\textless{}{-}}\NormalTok{ iris[trainIndex, ]}
\NormalTok{testData }\OtherTok{\textless{}{-}}\NormalTok{ iris[}\SpecialCharTok{{-}}\NormalTok{trainIndex, ]}
\end{Highlighting}
\end{Shaded}

~

Uma etapa simples de seleção de variáveis é exemplificada no bloco a
seguir. A função \texttt{rfeControl} define como será feita a validação
--- aqui é feita validação cruzada com 10 partições da base de dados com
tamanhos similares.

~

\begin{Shaded}
\begin{Highlighting}[]
\NormalTok{control }\OtherTok{\textless{}{-}} \FunctionTok{trainControl}\NormalTok{(}\AttributeTok{method =} \StringTok{"cv"}\NormalTok{, }\AttributeTok{number =} \DecValTok{10}\NormalTok{)}

\CommentTok{\# Train logistic regression model}
\NormalTok{logistic\_model }\OtherTok{\textless{}{-}} \FunctionTok{train}\NormalTok{(}
\NormalTok{  Class }\SpecialCharTok{\textasciitilde{}}\NormalTok{ .,                            }\CommentTok{\# Formula for the model}
  \AttributeTok{data =}\NormalTok{ trainData,                     }\CommentTok{\# Training data}
  \AttributeTok{method =} \StringTok{"glm"}\NormalTok{,                       }\CommentTok{\# Method: Generalized Linear Model}
  \AttributeTok{family =} \StringTok{"binomial"}\NormalTok{,                  }\CommentTok{\# Family: Binomial (logistic regression)}
  \AttributeTok{trControl =}\NormalTok{ control                   }\CommentTok{\# Cross{-}validation control parameters}
\NormalTok{)}
\end{Highlighting}
\end{Shaded}

~

Os coeficientes, seus desvios e significâncias individuais são dados a
seguir:

\begin{Shaded}
\begin{Highlighting}[]
\NormalTok{coefficients }\OtherTok{\textless{}{-}} \FunctionTok{summary}\NormalTok{(logistic\_model}\SpecialCharTok{$}\NormalTok{finalModel)}\SpecialCharTok{$}\NormalTok{coefficients}
\FunctionTok{print}\NormalTok{(coefficients)}
\end{Highlighting}
\end{Shaded}

\begin{verbatim}
                       Estimate Std. Error       z value  Pr(>|z|)
(Intercept)       -2.656607e+01   419598.9 -6.331301e-05 0.9999495
Sepal.Length       7.255362e-12   104971.0  6.911776e-17 1.0000000
Sepal.Width       -5.746443e-12   119016.7 -4.828264e-17 1.0000000
Petal.Length      -8.271818e-12   121523.0 -6.806794e-17 1.0000000
Petal.Width       -1.904911e-12   187285.1 -1.017118e-17 1.0000000
Speciesversicolor  5.313214e+01   312446.2  1.700521e-04 0.9998643
Speciesvirginica   2.286765e-11   430478.6  5.312147e-17 1.0000000
\end{verbatim}

~

Usando o mesmo pacote, a seleção de variáveis poderia ser feita da
seguinte forma, utilizando o método de seleção
\texttt{recursive\ feature\ elimination}:

~

\begin{Shaded}
\begin{Highlighting}[]
\NormalTok{ctrl }\OtherTok{\textless{}{-}} \FunctionTok{rfeControl}\NormalTok{(}\AttributeTok{functions =}\NormalTok{ rfFuncs, }\AttributeTok{method =} \StringTok{"cv"}\NormalTok{, }\AttributeTok{number =} \DecValTok{10}\NormalTok{)}

\NormalTok{rfe\_model }\OtherTok{\textless{}{-}} \FunctionTok{rfe}\NormalTok{(}
\NormalTok{  dplyr}\SpecialCharTok{::}\FunctionTok{select}\NormalTok{(trainData, }\SpecialCharTok{{-}}\NormalTok{Species, }\SpecialCharTok{{-}}\NormalTok{Class),  }
\NormalTok{  trainData}\SpecialCharTok{$}\NormalTok{Class,                      }
  \AttributeTok{sizes =} \FunctionTok{c}\NormalTok{(}\DecValTok{1}\SpecialCharTok{:}\DecValTok{4}\NormalTok{),                       }
  \AttributeTok{rfeControl =}\NormalTok{ ctrl                    }
\NormalTok{)}

\NormalTok{rfe\_model}
\end{Highlighting}
\end{Shaded}

\begin{verbatim}

Recursive feature selection

Outer resampling method: Cross-Validated (10 fold) 

Resampling performance over subset size:

 Variables   RMSE Rsquared     MAE  RMSESD RsquaredSD   MAESD Selected
         1 0.2206   0.7447 0.09050 0.13513     0.2198 0.06772         
         2 0.1541   0.8505 0.05990 0.11741     0.1693 0.05415        *
         3 0.1687   0.8550 0.08088 0.09827     0.1550 0.04936         
         4 0.1829   0.8366 0.10219 0.08731     0.1332 0.05308         

The top 2 variables (out of 2):
   Petal.Length, Petal.Width
\end{verbatim}



\end{document}
